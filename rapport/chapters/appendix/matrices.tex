On se donne une suite de $x_k$ pour lesquels on cherche une sous-famille non vide telle que le produit des image des $x_i$ par le \hyperref[polynome de Kraitchik]{polynôme de Kraitchik} est un carré.\\

Le principe est de récupérer la décomposition en facteurs premiers des différents $x_k$ lors de notre \hyperref[B-friabilite]{recherche des entiers B-friables}. 
À partir de cela, on crée une matrice, où chaque colonne représente un  des $x_k$, chaque ligne est un des nombres premiers plus petits que B et, dans chaque case, va être inscrite la puissance du facteur modulo 2.\\

Une fois notre matrice créée, nous allons calculer son \hyperref[Noyau]{noyau}. 
Avec celui-ci, nous prenons un des vecteurs. Si à la position i il y a un 1 c'est que nous devons prendre le $x_i$ tandis que s'il y a un 0 c'est que nous n'en avons pas besoin.\\

Grâce à cela nous avons déterminé les entiers dont nous aurons besoin pour notre factorisation.\\

\begin{proposition}
    Nous sommes sûr de trouver une solution, si on a $k > \pi(B)$.
\end{proposition}

\begin{proof}
    Soit M, la matrice créée.\\
    Par le théorème du rang, on a \[dim(ker(M)) + dim(Im(M)) = k.\]
    D'où $\hspace{4.5cm}dim(ker(M)) = k - dim(Im(M))\vspace{2mm}$\\
    Or $\hspace{6cm}Im(M) \subseteq {(\mathbb{Z}/2\mathbb{Z})}^{\pi(B)}\vspace{2mm}$\\
    donc $\hspace{5cm}dim(Im(M)) \le \pi(B)\vspace{2mm}$\\
    Ainsi $\hspace{6cm}-\pi(B) \le - dim(Im(M))\vspace{2mm}$\\
    et $\hspace{6cm}k - \pi(B) \le k - dim(Im(M))\vspace{2mm}$\\
    Finalement $\hspace{4cm}dim(ker(M)) \geq k - \pi(B) > 0\vspace{2mm}$\\
    Il y a donc bien des solutions non triviales.
\end{proof}

Voici notre implémentation de la recherche de congruences carrées:
\begin{lstlisting}[language=Python]
def _find_2_divider(n, primes, k):
	(x_k, q_x_k) = _find_friables(n, primes, k)
	mat = _create_mat(q_x_k, primes)
	sols = resolution.resolution(mat)
	(fct1, fct2) = (1, 1)
	for sol in sols:
		u = []
		v2 = []
		for i in range(len(sol)):
			if sol[i] == 1:
				u.append(x_k[i])
				v2.append(q_x_k[x_k[i]])
		v = _sqrt_v(v2)
		(fct1, fct2) = kraitchik.Kraitchik(n, u, v)
		if (fct1 != 1 and fct1 != n) or (fct2 != 1 and fct2 != n):
			break
	fct1 = 1 if fct1 == 1 or fct1 == n else fct1
	fct2 = 1 if fct2 == 1 or fct2 == n else fct2
	return (fct1, fct2)
\end{lstlisting}
\vspace{1em}
La fonction \lstinline{_find_friables} renvoie un tableau contenant les entiers tels que leurs carrés sont B-friables et la décomposition en facteurs premiers de leurs carrés. \\
À partir de cette décomposition, nous pouvons créer la matrice, puis la résoudre.
Ensuite, on parcourt les différents vecteurs des solutions trouvées pour pouvoir en extraire des facteurs.

\begin{example}
    Prenons, n = 2041.\\ 
    Nous avons B = 7 donc primes = $[2, 3, 5, 7]$ et k = 5.\\
    On trouve six entiers 7-friables: $[46, 47, 49, 51, 53, 54]$
    avec la décomposition de leurs carrés: \[{46:\{3: 1, 5: 2\}, 47:\{2: 3, 3: 1, 7: 1\}, 49:\{2: 3, 3: 2, 5: 1\},51: \{2: 4, 5: 1, 7: 1\},53: \{2: 8, 3: 1\},54: \{5: 3, 7: 1\}}\]
    Avec ça nous en déduisons la matrice suivante, en mettant modulo 2 les puissances:
    \[\begin{pmatrix}
        0 & 1 & 1 & 0 & 0 & 0\\
        1 & 1 & 0 & 0 & 1 & 0\\
        0 & 0 & 1 & 1 & 0 & 1\\
        0 & 1 & 0 & 1 & 0 & 1
    \end{pmatrix}\]
    En calculant le noyau on trouve:
    \[\begin{pmatrix}
        1 & 1\\
        0 & 1\\
        0 & 1\\
        0 & 0\\
        1 & 0\\
        0 & 1
    \end{pmatrix}\]

    Nous avons donc deux possibilités pour trouver une factorisation.
    On peut soit utilser notre premier et notre cinquième entiers, soit utilisé notre premier, deuxième, troisième et sixième entiers
\end{example}