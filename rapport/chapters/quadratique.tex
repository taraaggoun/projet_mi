La principe du crible quadratique a été inventée par C. Pomerance en 1981. 
C’est aujourd’hui la méthode la plus couramment utilisée pour factoriser des entiers n qui n’ont pas de diviseurs premiers significativement plus petits que $\sqrt{n}$. 
Elle ne dépend en fait que de la taille de n, et non de propriétés arithmétiques particulières de ses diviseurs premiers.\\

Afin de factoriser n, on recherche des congruences modulo n entre carrés, c'est-à-dire que l'on cherche des entiers dont la différence de leurs carrés soit un multiple de n, et ne vaut pas n.
Supposons que l'on ait deux entiers u et v tels que: \[u^2 \equiv v^2 [n] ~ avec ~ u \not\equiv \pm v [n]\textrm{.}\]
Dans ce cas, on peut obtenir très simplement une factorisation de n, vu que n divise $(u - v)(u + v)$ sans diviser $u - v$ ni $u + v$. Le calcul des entiers \[pgcd(u - v, n) ~ et ~ pgcd(u + v, n)\] fournit alors des diviseurs non triviaux de n.\\

\begin{definition*}\label{polynome de Kraitchik}

On introduit alors le polynôme de Kraitchik, \[Q(X) = X^2 - n \in \mathbb{Z}[X]\]
    
\end{definition*}

\begin{definition*}[friabilité]\label{friable}
    Soit B un entier naturel. On dit qu'un entier est B-friable si tous ses diviseurs premiers sont inférieurs ou égaux à B.\\
    
\end{definition*}
Le but va être de trouver une suite d'entiers $x_1, ~ ..., ~ x_k$ avec  $k > \pi(B)$ tel que $Q(x_1), ~ ..., ~ Q(x_k)$ sont B-friables et 
\[Q(x_1) ~ ... ~ Q(x_k) = v^2 ~ ou ~ v\in \mathbb{N}\] et \[u = x_1 ~ ... ~ x_k.\]
On a alors \[u^2 \equiv (x_i^2 - n) ~ ... ~ (x_k^2 - n) = Q(x_1) ~ ... ~ Q(x_k) = v^2 ~ [n].\]
De plus, il y a une grande probabilité que $u \not\equiv \pm v$ \\

La difficulté est donc de trouver ces $x_k$. 
Pour cela, nous allons commencer par déterminer quelle valeur prendre pour notre B de la B-friabilité. Si B est trop petit, cela risque d'être compliqué de conclure car on n'est pas sûr de parvenir à expliciter un seul entier x qui soit B-friable. Si B est trop grand, il faudra un très grand nombre de x, et on n'est pas sûr d'en trouver assez. Il faut trouver un juste milieu.
Le comportement asymptotique de B en fonction de n incite à choisir: \[B = exp(\frac{1}{2}\sqrt{log(n)log(log(n))})\]

Une fois que l'on a déterminé B, on cherche k entiers B-friables grâce à la \hyperref[B-friabilite]{méthode de recherche des entiers B-friables}.\\

Une fois que l'on a trouvé notre suite $x_k$ d'entiers, il va falloir choisir lesquels d'entre eux nous seront utiles pour la factorisation de n. Pour cela, nous allons appliquer le principe de la \hyperref[Matrices]{recherche de congruences de carrés}.\\

\begin{proof}
    Notons $p_1, ~ ..., ~ p_{\pi(B)}$ les nombres premiers inférieurs ou égaux à B.\\
    $\forall ~ 1 \leq j \leq k$, on a,
    \[Q(x_j) = \prod\limits_{i=1}^{\pi(B)} {p_i}^{n_{ij}} \textrm{, avec } n_{ij} \in \mathbb{N}\]
    on pose
    \begin{align*}
        \prod\limits_{j=1}^{k} Q(x_j) = \prod\limits_{j=1}^k {\prod\limits_{i=1}^{\pi(B)} {p_i}^{n_{ij}}}\\
        = \prod\limits_{i=1}^{\pi(B)} {{\prod\limits_{j=1}^k {p_i}^{n_{ij}}}}\\
        = \prod\limits_{i=1}^{\pi(B)} {{p_i}^{\sum\limits_{j=1}^k {n_{ij}}}}
    \end{align*}
 
    En particulier, $\prod\limits_{j=1}^{k} Q(x_j)$ est un carré ssi $\sum\limits_{i=1}^k {n_{ij}} \equiv 0 [2], ~ \forall ~ 1 \leq j \leq \pi(B)$\\
    
    ssi $\forall ~ 1 \leq j \leq \pi(B), ~ \sum\limits_{i=1}^k {\overline{n_{ij}}} = \overline{0}$, où $\overline{0}$ désigne la classe de 0 dans $\mathbb{Z}/2\mathbb{Z}$.
    
    Introduisons la matrice $M = (\overline{n_{ij}}) \in M_{\pi(B),k} ~ (\mathbb{Z}/2\mathbb{Z})$, la matrice contenant les puissances modulo 2 des $Q(x_i)$.\\
    Soit $X = 
    \begin{pmatrix}
        \alpha_1 \\ ... \\ \alpha_k
    \end{pmatrix}
    \in (\mathbb{Z}/2\mathbb{Z})^k$ tel que $X \in ker(M)$\\
    On a alors $\sum\limits_{j=1}^k {\alpha_j ~ \overline{n_{ij}}} = \overline{0}, ~ \forall ~ 1 \leq i \leq \pi(B)$.\\
    Notons $J = \left\{ j \in \left\{1, ..., k\right\} | \alpha_j = \overline{1} \right\}.$
    Alors \begin{align*}
        \sum\limits_{j=1}^k {\alpha_j \overline{n_{ij}}} = \overline{0} &\Longleftrightarrow \sum\limits_{j\in J} {\overline{n_{ij}}} = \overline{0} ~ \forall ~ 1 \leq i \leq \pi(B)\\
        &\Longleftrightarrow \sum\limits_{j\in J} {n_{ij}} ~ est ~ pair ~ \forall ~ 1 \leq i \leq \pi(B)\\
        &\Longleftrightarrow ~ \forall ~ 1 \leq i \leq \pi(B), ~ \exists ~ q_i \in \mathbb{N} ~ tq ~ \sum\limits_{j \in J} {n_{ij}} = 2 q_i
    \end{align*}
    Posons $u_J = \prod\limits_{j \in J} {x_j}$ et $v_J = \prod\limits_{i=1}^{\pi(B)} {{p_i}^{q_i}}$.\\
    On a alors 
    \begin{align*}
        {u_J}^2 = \prod\limits_{j\in J} {{x_j}^2} &\equiv \prod\limits_{j\in J} {Q(x_j) [n]}\\
        &\equiv \prod\limits_{i=1}^{\pi(B)} {{p_i}^{\sum\limits_{j \in J} {n_{ij}}} [n]}\\
        &\equiv \prod\limits_{i=1}^{\pi(B)} {{p_i}^{2q_i} [n]}\\
        &\equiv (\prod\limits_{i=1}^{\pi(B)} {{p_i}^{q_i})^2 [n]}\\
        &\equiv {v_J}^2 [n]
    \end{align*}
    Ainsi, à tout élément $X \in ker(M)$, on est en mesure d'associer deux éléments $u_J$ et $v_J$ tq ${u_J}^2 \equiv {v_J}^2 [n],$\\
    et qui ont une chance d'être tels que $u_J \not\equiv v_J [n]$
\end{proof}

Une fois que l'on a déterminé lesquels de nos $x_k$ nous seront utiles, on a enfin trouvé une factorisation de n, car nous avons \[v = \sqrt{Q(x_1) ~ ... ~ Q(x_k)}\] ainsi que: \[u = x_1 ~ ... ~ x_k\]
par ailleurs, au moins un de: \[pgcd(u - v, n) ~ et ~ pgcd(u + v, n)\] est un facteur de n. \\

Voici notre implémentation:\\

Nous avons une fonction \lstinline{_divider} qui prend en paramètre \lstinline{n}, le nombre à factoriser, et qui va chercher deux diviseurs non triviaux de n. 
Cette fonction utilise \lstinline{find_b} qui renvoie le B adéquat en fonction de n, \lstinline{eratosthene}, qui utilise \hyperref[Eratosthène]{le crible d'Eratosthène} pour trouver les nombres premiers inférieurs à B, et \lstinline{_find_2_divider} qui est expliqué dans \hyperref[Matrices]{l'annexe C}.
\clearpage
\begin{lstlisting}[language=Python]
def _divider(n):
    b = utils.find_b(n)
    primes = ce.eratosthene(b)
    k = len(primes) + 1
    (fct1, fct2) = (1, 1)
    while(1):
        (fct1, fct2) = _find_2_divider(n, primes, k)
	if fct1 != 1 or fct2 != 1:
		break
	k += 1
    return (fct1, fct2)
\end{lstlisting}
\vspace{1em}

\begin{example}
    Essayons de factoriser 2041.\\
    On commence par chercher B. On trouve B = 7.\\
    et tous les nombres premiers inférieurs ou égaux à B sont
    \[[2, 3, 5, 7]\]
    On cherche donc k > $\pi(7) = 4$. \\
    On trouve donc grâçe à la \hyperref[B-friabilite]{recherche des entiers B-Friables} 6 entiers, dont leurs images par le \hyperref[polynome de Kraitchik]{polynôme de Kraitchik} sont 7-friables.
    \[[46, 47, 49, 51, 53, 54]\]
    avec leurs images
    \[75 = 3 \times 5^2, ~ 168 = 2^3 \times 3 \times 7, ~ 360 = 2^3 \times 3^2 \times 5, ~ 560 = 2^4 \times 5 \times 7, ~ 768 = 2^8 \times 3, ~ 875 = 5^3 \times 7\]

    À partir de ces 6 entiers et de la \hyperref[Matrices]{recherche de congruences de carrés.}
    On trouve cette matrice

    \[\begin{pmatrix}
    0 & 1 & 1 & 0 & 0 & 0\\
    1 & 1 & 0 & 0 & 1 & 0\\
    0 & 0 & 1 & 1 & 0 & 1\\
    0 & 1 & 0 & 1 & 0 & 1
    \end{pmatrix}\]

    dont le noyau est engendré par
    \[\begin{pmatrix} 1\\ 0\\ 0\\ 0\\ 1\\ 0\end{pmatrix} \textrm{ , } \begin{pmatrix} 1\\ 1\\ 1\\ 0\\ 0\\ 1\end{pmatrix}\textrm{.}\]

    Nous allons utiliser le premier vecteur, On doit ainsi utiliser notre premier entier qui est 46 et le 5eme qui est 53.
    On a \[u = 46 \times 53 = 2438\]
    et \[v^2 = 75 \times 768 = 2^8 \times 3^2 \times 5^2\]
    Donc \[v = 2^4 \times 3 \times 5 = 240\]
    on a ainsi \[u + v = 2678 ~ et ~ u - v = 2198\]
    on calcule \[pgcd(2678, 2041) = 13 ~ et ~ pgcd(2198, 2041) = 157\]
    on en conclut que \[2041 = 13 \times 157\]
    Avec le \hyperref[Miller_Rabin]{test de primalité Miller Rabin} on trouve que 13 et 157 sont premiers, on a donc fini.
\end{example}
