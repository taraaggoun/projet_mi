La méthode de Fermat a été trouvée par le mathématicien français Pierre de Fermat.
Elle est très efficace pour factoriser n, si n s’écrit comme un produit de deux entiers proches l’un de l’autre.
Elle repose sur le fait que déterminer une factorisation de n est équivalent à écrire n comme une différence de deux carrés. \\

Le principe de cette méthode est de trouver un u tel que $(\lfloor \sqrt{n} \rfloor + u) ^ 2 - n$ 
est un carré que l’on va noter $s^2$. 
On note aussi $r = \lfloor \sqrt{n} \rfloor + u$.
Une fois trouvés notre r et notre s, on va pouvoir factoriser n car on a :
       \[r^2 - n = s^2 ~~ \Leftrightarrow ~~ n = (r - s) (r + s)\] \\
Afin de déterminer u, on examine successivement les entiers 
\[\lfloor \sqrt{n} \rfloor + 1, ~ \lfloor \sqrt{n} \rfloor + 2, ~ ...\]

Voici notre implémentation de la méthode de Fermat:
\begin{lstlisting}[language=Python]
def _find_a_b(n): 
    (b, root) = _is_square(n)
    if b == 1: 
        return (root, root)
    for u in range (1, ((n + 1) // 2) + 1):
        r = root + u
        (b, s) = _is_square((r * r) - n)
        if b == 1: 
            return (r - s, r + s)
    return (n, 1)

def all_divider(n, facteurs={}):
    (a, b) = _find_a_b(n)
    if a == n or b == n:
        facteurs[n] = facteurs.get(n, 0) + 1
    else:
        utils.merge_dicts(facteurs, all_divider(a, facteurs))
        utils.merge_dicts(facteurs, all_divider(b, facteurs))
    return facteurs
\end{lstlisting}
\vspace{1em}

\newpage
Nous avons ici deux fonctions, la première, \lstinline{_find_a_b} qui prend en paramètre le nombre qu'on veut factoriser, \lstinline{n}. 
Cette fonction va retourner deux facteurs de n en utilisant la méthode de Fermat.
La deuxième fonction \lstinline{_all_divider} qui prend en arguments \lstinline{n}, le nombre à factoriser, et \lstinline{facteurs}, le dictionnaire qui va contenir tous les facteurs de n. Elle va appeler \lstinline{_find_a_b} qui permet de trouver deux facteurs et va s'appeler récursivement pour chercher tous les facteurs.
De plus, une fonction auxiliaire est appelée, \lstinline{_is_square(n)}, qui renvoie un couple, la première valeur vaut 1 si n est un carré parfait et la seconde est sa racine carrée; sinon elle renvoie -1 et la valeur absolue de sa racine carrée.

\begin{example}
   Grâce à cette méthode, nous parvenons à factoriser n = 399772198733 en moins de deux secondes.
   En effet nous commençons par vérifier que n n'est pas un carré parfait. \\ 
   On a que \[\lfloor \sqrt{n} \rfloor = 632275\]
    Ensuite on itère sur u jusqu'à ce que $(\lfloor \sqrt{n} \rfloor + u) ^ 2 - n$, soit un carré, ce qui arrive pour  \[u = 4022\]
    On a donc \[r = 636297 ~ \textrm{et} ~ s = 71426 \textrm{,}\]
    et on obtient \[399772198733 = 564871 \times 707723\]

    Maintenant, nous allons recommencer cette procédure sur les deux facteurs:\\
    Pour n = 564871, on trouve :
    \[u = 281685\]
    \[r = 282436 ~ et ~ s = 282435\]
    Donc \[564871 = 1 \times  564871\]
    Ainsi 564871 est premier. \\

    De la même façon, pour n = 707723, on trouve :
    \[u = 353021\]
    \[r = 353862 ~ et ~ s = 353861 \textrm{.}\]
    Donc \[707723 = 1 \times  707723\textrm{,}\]
    ce qui prouve que 707723 est premier. \\

    On a bien à la fin que \[399772198733 = 564871 \times 707723\]
    
\end{example}